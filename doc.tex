\documentclass{article}\usepackage{graphicx, color}
%% maxwidth is the original width if it is less than linewidth
%% otherwise use linewidth (to make sure the graphics do not exceed the margin)
\makeatletter
\def\maxwidth{ %
  \ifdim\Gin@nat@width>\linewidth
    \linewidth
  \else
    \Gin@nat@width
  \fi
}
\makeatother

\IfFileExists{upquote.sty}{\usepackage{upquote}}{}
\definecolor{fgcolor}{rgb}{0.2, 0.2, 0.2}
\newcommand{\hlnumber}[1]{\textcolor[rgb]{0,0,0}{#1}}%
\newcommand{\hlfunctioncall}[1]{\textcolor[rgb]{0.501960784313725,0,0.329411764705882}{\textbf{#1}}}%
\newcommand{\hlstring}[1]{\textcolor[rgb]{0.6,0.6,1}{#1}}%
\newcommand{\hlkeyword}[1]{\textcolor[rgb]{0,0,0}{\textbf{#1}}}%
\newcommand{\hlargument}[1]{\textcolor[rgb]{0.690196078431373,0.250980392156863,0.0196078431372549}{#1}}%
\newcommand{\hlcomment}[1]{\textcolor[rgb]{0.180392156862745,0.6,0.341176470588235}{#1}}%
\newcommand{\hlroxygencomment}[1]{\textcolor[rgb]{0.43921568627451,0.47843137254902,0.701960784313725}{#1}}%
\newcommand{\hlformalargs}[1]{\textcolor[rgb]{0.690196078431373,0.250980392156863,0.0196078431372549}{#1}}%
\newcommand{\hleqformalargs}[1]{\textcolor[rgb]{0.690196078431373,0.250980392156863,0.0196078431372549}{#1}}%
\newcommand{\hlassignement}[1]{\textcolor[rgb]{0,0,0}{\textbf{#1}}}%
\newcommand{\hlpackage}[1]{\textcolor[rgb]{0.588235294117647,0.709803921568627,0.145098039215686}{#1}}%
\newcommand{\hlslot}[1]{\textit{#1}}%
\newcommand{\hlsymbol}[1]{\textcolor[rgb]{0,0,0}{#1}}%
\newcommand{\hlprompt}[1]{\textcolor[rgb]{0.2,0.2,0.2}{#1}}%

\usepackage{framed}
\makeatletter
\newenvironment{kframe}{%
 \def\at@end@of@kframe{}%
 \ifinner\ifhmode%
  \def\at@end@of@kframe{\end{minipage}}%
  \begin{minipage}{\columnwidth}%
 \fi\fi%
 \def\FrameCommand##1{\hskip\@totalleftmargin \hskip-\fboxsep
 \colorbox{shadecolor}{##1}\hskip-\fboxsep
     % There is no \\@totalrightmargin, so:
     \hskip-\linewidth \hskip-\@totalleftmargin \hskip\columnwidth}%
 \MakeFramed {\advance\hsize-\width
   \@totalleftmargin\z@ \linewidth\hsize
   \@setminipage}}%
 {\par\unskip\endMakeFramed%
 \at@end@of@kframe}
\makeatother

\definecolor{shadecolor}{rgb}{.97, .97, .97}
\definecolor{messagecolor}{rgb}{0, 0, 0}
\definecolor{warningcolor}{rgb}{1, 0, 1}
\definecolor{errorcolor}{rgb}{1, 0, 0}
\newenvironment{knitrout}{}{} % an empty environment to be redefined in TeX

\usepackage{alltt}
\usepackage[hidelinks]{hyperref}
\begin{document}
This will be a momentum strategy.  Much of this will come from \href{http://rbresearch.wordpress.com/2012/08/23/momentum-with-r-part-1/}{Ross Bennett's blog}.  
\section{Preparation}
\begin{knitrout}
\definecolor{shadecolor}{rgb}{0.969, 0.969, 0.969}\color{fgcolor}\begin{kframe}
\begin{alltt}
\hlfunctioncall{require}(quantstrat)
\hlfunctioncall{currency}(\hlstring{"USD"})
\end{alltt}
\begin{verbatim}
## [1] "USD"
\end{verbatim}
\begin{alltt}
symbols = \hlfunctioncall{c}(\hlstring{"XLY"}, \hlstring{"XLP"}, \hlstring{"XLE"}, \hlstring{"XLF"})
\hlfunctioncall{stock}(symbols, currency = \hlstring{"USD"}, multiplier = 1)
\end{alltt}
\begin{verbatim}
## [1] "XLY" "XLP" "XLE" "XLF"
\end{verbatim}
\begin{alltt}
\hlfunctioncall{getSymbols}(symbols, src = \hlstring{"yahoo"}, index.class = \hlfunctioncall{c}(\hlstring{"POSIXt"}, \hlstring{"POSIXct"}), from = \hlstring{"2000-01-01"})
\end{alltt}
\begin{verbatim}
## [1] "XLY" "XLP" "XLE" "XLF"
\end{verbatim}
\end{kframe}
\end{knitrout}

Convert to monthly and drop all columns except Adjusted Close
\begin{knitrout}
\definecolor{shadecolor}{rgb}{0.969, 0.969, 0.969}\color{fgcolor}\begin{kframe}
\begin{alltt}
\hlfunctioncall{for} (symbol in symbols) \{
    x <- \hlfunctioncall{get}(symbol)
    x <- \hlfunctioncall{to.monthly}(x, indexAt = \hlstring{"lastof"}, drop.time = TRUE)
    \hlfunctioncall{indexFormat}(x) <- \hlstring{"%Y-%m-%d"}
    \hlfunctioncall{colnames}(x) <- \hlfunctioncall{gsub}(\hlstring{"x"}, symbol, \hlfunctioncall{colnames}(x))
    x <- x[, 6]  \hlcomment{#drops all columns except Adjusted Close which is 6th column}
    \hlfunctioncall{assign}(symbol, x)
\}
\end{alltt}
\end{kframe}
\end{knitrout}

There are now four objects that have an adjusted close.Merge the symbols into a single object with just the close prices
\begin{knitrout}
\definecolor{shadecolor}{rgb}{0.969, 0.969, 0.969}\color{fgcolor}\begin{kframe}
\begin{alltt}
symbols_close <- \hlfunctioncall{do.call}(merge, \hlfunctioncall{lapply}(symbols, get))
\hlfunctioncall{head}(symbols_close)
\end{alltt}
\begin{verbatim}
##            XLY.Adjusted XLP.Adjusted XLE.Adjusted XLF.Adjusted
## 2000-01-31        23.75        17.94        22.57        17.84
## 2000-02-29        22.43        15.84        21.62        15.93
## 2000-03-29        25.56        16.39        24.22        18.78
## 2000-04-28        25.03        17.25        23.86        18.96
## 2000-05-29        23.67        18.49        26.67        19.38
## 2000-06-28        22.39        19.53        25.19        18.44
\end{verbatim}
\end{kframe}
\end{knitrout}

Now calculate the three period rate of change so that the returns can be ranked to see where the momentum lies. 
\begin{knitrout}
\definecolor{shadecolor}{rgb}{0.969, 0.969, 0.969}\color{fgcolor}\begin{kframe}
\begin{alltt}
roc <- \hlfunctioncall{ROC}(symbols_close, n = 3, type = \hlstring{"discrete"})
\hlfunctioncall{head}(roc)
\end{alltt}
\begin{verbatim}
##            XLY.Adjusted XLP.Adjusted XLE.Adjusted XLF.Adjusted
## 2000-01-31           NA           NA           NA           NA
## 2000-02-29           NA           NA           NA           NA
## 2000-03-29           NA           NA           NA           NA
## 2000-04-28      0.05389     -0.03846      0.05716      0.06278
## 2000-05-29      0.05528      0.16730      0.23358      0.21657
## 2000-06-28     -0.12402      0.19158      0.04005     -0.01810
\end{verbatim}
\end{kframe}
\end{knitrout}

Now apply the rank function across each column. The symbol with the highest return has the rank one for each month. This will be an xts object with ranks. 
\begin{knitrout}
\definecolor{shadecolor}{rgb}{0.969, 0.969, 0.969}\color{fgcolor}\begin{kframe}
\begin{alltt}
r <- \hlfunctioncall{as.xts}(\hlfunctioncall{t}(\hlfunctioncall{apply}(-roc, 1, rank)))
\hlfunctioncall{head}(r)
\end{alltt}
\begin{verbatim}
##            XLY.Adjusted XLP.Adjusted XLE.Adjusted XLF.Adjusted
## 2000-01-31            1            2            3            4
## 2000-02-29            1            2            3            4
## 2000-03-29            1            2            3            4
## 2000-04-28            3            4            2            1
## 2000-05-29            4            3            1            2
## 2000-06-28            4            1            2            3
\end{verbatim}
\end{kframe}
\end{knitrout}

\section{Functions}
\subsection{RankingRB}
Computes the rank of an xts object of ranking factors.  Ranking factors are the factors that are ranked (i.e. asset returns)
\begin{itemize}
\item{args}
  x = xts object of ranking factors
  
\item{Returns}
  Returns an xts object with ranks
  (e.g. for ranking asset returns, the asset with the greatest return
  receives a  rank of 1)
\end{itemize}
\begin{knitrout}
\definecolor{shadecolor}{rgb}{0.969, 0.969, 0.969}\color{fgcolor}\begin{kframe}
\begin{alltt}
RankRB <- \hlfunctioncall{function}(x) \{
    r <- \hlfunctioncall{as.xts}(\hlfunctioncall{t}(\hlfunctioncall{apply}(-x, 1, rank, na.last = \hlstring{"keep"})))
    \hlfunctioncall{return}(r)
\}
\end{alltt}
\end{kframe}
\end{knitrout}

\subsection{MontlyAd}
   Converts daily data to monthly and returns only the monthly close. 
\begin{itemize}
\item Args:
  x = daily price data from Yahoo Finance
  
\item{Returns}
  xts object with the monthly adjusted close prices
\end{itemize}
\begin{knitrout}
\definecolor{shadecolor}{rgb}{0.969, 0.969, 0.969}\color{fgcolor}\begin{kframe}
\begin{alltt}
MonthlyAd <- \hlfunctioncall{function}(x) \{
    sym <- \hlfunctioncall{sub}(\hlstring{"\textbackslash{}\textbackslash{}..*$"}, \hlstring{""}, \hlfunctioncall{names}(x)[1])
    \hlfunctioncall{Ad}(\hlfunctioncall{to.monthly}(x, indexAt = \hlstring{"lastof"}, drop.time = TRUE, name = sym))
\}
\end{alltt}
\end{kframe}
\end{knitrout}

\subsection{CAGR}
Function to compute the CAGR given simple returns
\begin{itemize}
\item Args:
  x = xts of simple returns
  m = periods per year (i.e. monthly = 12, daily = 252)
\item Returns the Compound Annual Growth Rate
\end{itemize}  
\begin{knitrout}
\definecolor{shadecolor}{rgb}{0.969, 0.969, 0.969}\color{fgcolor}\begin{kframe}
\begin{alltt}
CAGR <- \hlfunctioncall{function}(x, m) \{
    x <- \hlfunctioncall{na.omit}(x)
    cagr <- \hlfunctioncall{apply}(x, 2, \hlfunctioncall{function}(x, m) \hlfunctioncall{prod}(1 + x)^(1/(\hlfunctioncall{length}(x)/m)) - 1, m = m)
    \hlfunctioncall{return}(cagr)
\}
\end{alltt}
\end{kframe}
\end{knitrout}

\subsection{SimpleMomentumTest}
Returns a list containing a matrix of individual asset returns and the comnbined returns. Trade the top n asset(s) if the rank of last period is less than or equal to n,then I would experience the return for this month.
\begin{itemize}
\item args:
  xts.ret = xts of one period returns
  xts.rank = xts of ranks
  n = number of top ranked assets to trade
  ret.fill.na = number of return periods to fill with NA
\item Returns:
  An xts object of simple returns
\end{itemize}
\begin{knitrout}
\definecolor{shadecolor}{rgb}{0.969, 0.969, 0.969}\color{fgcolor}\begin{kframe}
\begin{alltt}
SimpleMomentumTest <- \hlfunctioncall{function}(xts.ret, xts.rank, n = 1, ret.fill.na = 3) \{
\hlcomment{    # returns a list containing a matrix of individual asset returns and the}
\hlcomment{    # comnbined returns args: xts.ret = xts of one period returns xts.rank =}
\hlcomment{    # xts of ranks n = number of top ranked assets to trade ret.fill.na =}
\hlcomment{    # number of return periods to fill with NA}
\hlcomment{    # }
\hlcomment{    # Returns: returns an xts object of simple returns}
    
\hlcomment{    # trade the top n asset(s) if the rank of last period is less than or}
\hlcomment{    # equal to n, then I would experience the return for this month.}
    
\hlcomment{    # lag the rank object by one period to avoid look ahead bias}
    lag.rank <- \hlfunctioncall{lag}(xts.rank, k = 1, na.pad = TRUE)
    n2 <- \hlfunctioncall{nrow}(lag.rank[\hlfunctioncall{is.na}(lag.rank[, 1]) == TRUE])
    z <- \hlfunctioncall{max}(n2, ret.fill.na)
    
\hlcomment{    # for trading the top ranked asset, replace all ranks above n with NA to}
\hlcomment{    # set up for element wise multiplication to get the realized returns}
    lag.rank <- \hlfunctioncall{as.matrix}(lag.rank)
    lag.rank[lag.rank > n] <- NA
\hlcomment{    # set the element to 1 for assets ranked <= to rank}
    lag.rank[lag.rank <= n] <- 1
    
\hlcomment{    # element wise multiplication of the 1 period return matrix and lagged}
\hlcomment{    # rank matrix}
    mat.ret <- \hlfunctioncall{as.matrix}(xts.ret) * lag.rank
    
\hlcomment{    # average the rows of the mat.ret to get the return for that period}
    vec.ret <- \hlfunctioncall{rowMeans}(mat.ret, na.rm = TRUE)
    vec.ret[1:z] <- NA
    
\hlcomment{    # convert to an xts object}
    vec.ret <- \hlfunctioncall{xts}(x = vec.ret, order.by = \hlfunctioncall{index}(xts.ret))
    f <- \hlfunctioncall{list}(mat = mat.ret, ret = vec.ret, rank = lag.rank)
    \hlfunctioncall{return}(f)
\}
\end{alltt}
\end{kframe}
\end{knitrout}


\end{document}
